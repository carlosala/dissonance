\documentclass{standalone}
\usepackage{tikz}
\usepackage{mathtools}
\usepackage{pgfplots, pgfplotstable}

\pgfplotsset{compat = newest}
\definecolor{lessgreen}{RGB}{0,100,0}

\begin{document}
    \begin{tikzpicture}
        \begin{axis}[
            xmin = 1, xmax = 2, %
            ymin = 0, ymax = 3.5, %
            xtick distance = 0.1, %Is the distance between major ticks in the x-axis.
            ytick distance = 0.5, %Is the distance between major ticks in the y-axis.
            grid = major, %When this options is set to both the minor and major grid are plotted.
            minor tick num = 1, %Is the number of ticks between major ticks.
            major grid style = {lightgray}, %Changes the color and stroke of the major grid.
            width = 10cm, %sets the width of the figure
            height = 7.5cm,  %sets the height of the figure
            xlabel = {$f_2/f_1$}, %
            ylabel = {$\mathcal{D}(\mathcal{S}_1,\mathcal{S}_2)$}, %
            legend cell align = {left}, %
        ]
            \IfStandalone{% options when file is compiled on its own
                \addplot [thick,smooth,no markers,red] table {./../Dades_i_codi/complex440.txt}; % other options to pass in options on []: col sep=comma,only marks, mark=oplus*,mark size=1pt
            }{% options when file is compiled from main.tex
                \addplot [thick,smooth,no markers,red] table {./Dades_i_codi/complex440.txt}; % other options to pass in options on []: col sep=comma,only marks, mark=oplus*,mark size=1pt
            }
            
            %%%%%%%% nota1 %%%%%%%% 
            \draw [thin,lessgreen,dashed] (axis cs:1.4983,0) -- node[right,pos=1.05]{659.26 Hz} (axis cs:1.4983,0.4498);
            \filldraw[lessgreen] (1.4983,0.4498) circle [radius=1.5pt];
            %%%%%%%% nota2 %%%%%%%% 
            \draw [thin,lessgreen,dashed] (axis cs:1.925,0) -- node[left,pos=1.05]{847 Hz} (axis cs:1.925,1.6140);
            \filldraw[lessgreen] (1.925,1.6140) circle [radius=1.5pt];
            \legend{$f_1=440\text{ Hz}$}
        \end{axis}
    \end{tikzpicture}
\end{document}